%%%%%%%%%%%%%%%%%%%%%%%%%%%%%%%%%%%%%%%%%
% "ModernCV" CV and Cover Letter
% LaTeX Template
% Version 1.1 (9/12/12)
%
% This template has been downloaded from:
% http://www.LaTeXTemplates.com
%
% Original author:
% Xavier Danaux (xdanaux@gmail.com)
%
% License:
% CC BY-NC-SA 3.0 (http://creativecommons.org/licenses/by-nc-sa/3.0/)
%
% Important note:
% This template requires the moderncv.cls and .sty files to be in the same
% directory as this .tex file. These files provide the resume style and themes
% used for structuring the document.
%
%%%%%%%%%%%%%%%%%%%%%%%%%%%%%%%%%%%%%%%%%

%----------------------------------------------------------------------------------------
%	PACKAGES AND OTHER DOCUMENT CONFIGURATIONS
%----------------------------------------------------------------------------------------

\documentclass[10pt,a4paper,roman]{moderncv} % Font sizes: 10, 11, or 12; paper sizes: a4paper, letterpaper, a5paper, legalpaper, executivepaper or landscape; font families: sans or roman

\moderncvstyle{banking} % CV theme - options include: 'casual' (default), 'classic', 'oldstyle' and 'banking'
\moderncvcolor{black} % CV color - options include: 'blue' (default), 'orange', 'green', 'red', 'purple', 'grey' and 'black'
\usepackage[UTF8]{ctex}
%\usepackage{utf8x}{inputenc}
\usepackage{url}
\usepackage{lipsum} % Used for inserting dummy 'Lorem ipsum' text into the template
\usepackage[scale=0.8]{geometry} % Reduce document margins
%sugartom_scale=0.75->0.8
%add the following line by sugartom
\geometry{top=1.5cm,bottom=2.0cm,head=0.5cm}
%\setlength{\hintscolumnwidth}{3cm} % Uncomment to change the width of the dates column
%\setlength{\makecvtitlenamewidth}{10cm} % For the 'classic' style, uncomment to adjust the width of the space allocated to your name

%----------------------------------------------------------------------------------------
%	NAME AND CONTACT INFORMATION SECTION
%----------------------------------------------------------------------------------------
\firstname{Gaoxiong} % Your first name
\familyname{Zeng} % Your last name

%----------------------------------------------------------------------------------------
\begin{document}

\makecvtitle % Print the CV title

%----------------------------------------------------------------------------------------
%	CONTACT SECTION
%----------------------------------------------------------------------------------------
\section{Contact Information}
\vspace{2mm}
\begin{tabular}{p{0.547\textwidth} r}
Homepage: \textcolor{blue}{\url{https://gaoxiongzeng.github.io}} & Office 3661 (Lift 31/32), Academic Building\\
Email: \href{mailto:gzengaa@cse.ust.hk}{gzengaa@cse.ust.hk}  / \href{mailto:gaoxiongzeng@qq.com}{gaoxiongzeng@qq.com} &  Hong Kong University of Science and Technology\\
Telephone: +852-68713173 / +86-14715683173 & Clear Water Bay, Kowloon, Hong Kong, China\\
\end{tabular}


%----------------------------------------------------------------------------------------
%	BIO SECTION
%----------------------------------------------------------------------------------------
\section{Short Biography}
\vspace{2mm}
Gaoxiong Zeng (曾高雄) is a Ph.D. student in computer science at Hong Kong University of Science and Technology (HKUST), advised by Prof.~Kai~Chen. He received his B.E. degree in electronic engineering from University of Science and Technology of China (USTC) in 2015. His research areas of interest include computer networks and systems, with special focuses on data center networking, and transport protocols.

%----------------------------------------------------------------------------------------
%	EDUCATION SECTION
%----------------------------------------------------------------------------------------
\section{Education}
\vspace{2mm}
\cventry{2015 - Present}{Ph.D. in Computer Science and Engineering}{Hong Kong University of Science and Technology (HKUST)}{Hong Kong, China}{}{Advisor: Prof. Kai Chen}
\vspace{2mm}
\cventry{2011 - 2015}{B.E. in Electronic Engineering and Information Science}{University of Science and Technology of China (USTC)}{Hefei, China}{}{}


%----------------------------------------------------------------------------------------
%	EXPERIENCE SECTION
%----------------------------------------------------------------------------------------
\section{Work Experiences}
\vspace{2mm}
\cvitemwithcomment{}{[Research Intern] {Peng Cheng Laboratory, Shenzhen, China.}}{2019.09 - 2020.01}
\vspace{1mm}
\cvitemwithcomment{}{[Research Intern] {Huawei Technologies Co., Ltd, Hong Kong, China.}}{2018.11 - 2019.01}
\vspace{1mm}
\cvitemwithcomment{}{[Research Intern] {Chinese Academy of Sciences, Beijing, China.}}{2014.07 - 2014.08}


%----------------------------------------------------------------------------------------
%	PUBLICATIONS SECTION
%----------------------------------------------------------------------------------------
\section{Publications}
\vspace{2mm}
\cvitem{Conference Publications}{
\vspace{1mm}
\begin{itemize}
\item Shuihai Hu, Wei Bai, \textbf{Gaoxiong Zeng}, Zilong Wang, Kai Chen, Kun Tan, Yi Wang. Aeolus: A Building Block for Proactive Transport in Datacenters, in Proceedings of the Annual Conference of the ACM Special Interest Group on Data Communication (\textbf{SIGCOMM}), Virtual Conference, August 10-14, 2020.
\vspace{1mm}
\item \textbf{Gaoxiong Zeng}, Wei Bai, Ge Chen, Kai Chen, Dongsu Han, Yibo Zhu, Lei Cui. Congestion Control for Cross-Datacenter Networks, in Proceedings of the 27th IEEE International Conference on Network Protocols (\textbf{ICNP}), Chicago, USA, October 7-10, 2019.
\end{itemize}
\vspace{1mm}
\begin{itemize}
\item Jiacheng Xia, \textbf{Gaoxiong Zeng}, Junxue Zhang, Weiyan Wang, Wei Bai, Junchen Jiang, Kai Chen. Rethinking Transport Layer Design for Distributed Machine Learning, in Proceedings of the 3rd ACM Asia-Pacific Workshop on Networking (\textbf{APNet}), Beijing, China, August 17-18, 2019.
\vspace{1mm}
\item \textbf{Gaoxiong Zeng}, Wei Bai, Ge Chen, Kai Chen, Dongsu Han, Yibo Zhu. Combining ECN and RTT for Datacenter Transport, in Proceedings of the 1st ACM Asia-Pacific Workshop on Networking (\textbf{APNet}), Hong Kong SAR, China, August
3-4, 2017.
\end{itemize}
}

\vspace{2mm}

\cvitem{Journal Publications}{
\vspace{1mm}
\begin{itemize}
\item \textbf{Gaoxiong Zeng}, Shuihai Hu, Junxue Zhang, Kai Chen. Transport Protocols for Data Center Networks: A Survey, in ICT-CAS/CCF Journal of Computer Research and Development (\textbf{J-CRAD}), 57(1), 2020.
\end{itemize}
\vspace{1mm}
\begin{itemize}
\item Wei Bai, Shuihai Hu, \textbf{Gaoxiong Zeng}, Kai Chen. Data Center Flow Scheduling, in Communications of the China Computer Federation (\textbf{CCCF}), 15(4), 2019.
\end{itemize}
}

\vspace{2mm}

\cvitem{Posters \& Demos}{
\vspace{1mm}
\begin{itemize}
\item Shuihai Hu, Kai Chen, \textbf{Gaoxiong Zeng}. Improved Path Compression for Explicit Path Control in Production Data Centers, in Poster Session of the 13th USENIX Symposium on Networked Systems Design and Implementation (\textbf{NSDI}),  Santa Clara, USA, March 16-18, 2016.
\end{itemize}
}


%----------------------------------------------------------------------------------------
%	TALKS SECTION
%----------------------------------------------------------------------------------------
\section{Selected Talks}
\vspace{2mm}
\cvitemwithcomment{}{Congestion Control for Cross-Datacenter Networks. ICNP 2019. Chicago, USA.}{Oct. 2019}
\vspace{1mm}
\cvitemwithcomment{}{Combining ECN and RTT for Datacenter Transport. APNet 2017. Hong Kong, China.}{Aug. 2017}

%----------------------------------------------------------------------------------------
%	SERVICES SECTION
%----------------------------------------------------------------------------------------
\section{Professional Activities}
\vspace{2mm}
\cvitemwithcomment{}{[Reviewer] INFOCOM 2017, etc.}{}
\vspace{1mm}
\cvitemwithcomment{}{[Review Drafter] NSDI 2021, INFOCOM 2021, CoNEXT 2020, SIGCOMM 2019, NSDI 2019, etc.}{}


%----------------------------------------------------------------------------------------
%	TEACHING SECTION
%----------------------------------------------------------------------------------------
\section{Teaching \& Tutoring}
\vspace{2mm}
\cvitemwithcomment{}{[Teaching Assistant] HKUST FYP: Transport Design for Data Center Networks}{Fall 2019, Spring 2020}
\vspace{1mm}
\cvitemwithcomment{}{[Teaching Assistant] HKUST COMP2611: Computer Organization}{Spring 2016, Fall 2016, Spring 2017}


%----------------------------------------------------------------------------------------
%	AWARDS SECTION
%----------------------------------------------------------------------------------------
\section{Honors \& Awards}
\vspace{2mm}
\cvitemwithcomment{}{[HKUST] {Research Postgraduate Scholarship}}{2015, 2016, 2017, 2018, 2019, 2020}
\vspace{1mm}
\cvitemwithcomment{}{[USTC] {Graduate with Honors of the Talent Program in Information Science and Technology}}{2015}
\vspace{1mm}
\cvitemwithcomment{}{[USTC] {Team Lead and 1$^{st}$ Place in Electronics Design Competition of the Talent Program}}{2014}
\vspace{1mm}
\cvitemwithcomment{}{[USTC] {Outstanding Student Scholarship}}{2012, 2013, 2014}
\vspace{1mm}
\cvitemwithcomment{}{[Department of Education, Guangdong] {Provincial Soong Ching Ling Scholarship} }{2010}


%----------------------------------------------------------------------------------------
%	NOTE SECTION
%----------------------------------------------------------------------------------------
\section{}
\begin{flushleft}
\textit{Last updated on Oct. 2020.}
\end{flushleft}

\end{document}
