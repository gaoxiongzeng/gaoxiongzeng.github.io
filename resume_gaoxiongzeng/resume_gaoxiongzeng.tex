%%%%%%%%%%%%%%%%%%%%%%%%%%%%%%%%%%%%%%%%%
% "ModernCV" CV and Cover Letter
% LaTeX Template
% Version 1.1 (9/12/12)
%
% This template has been downloaded from:
% http://www.LaTeXTemplates.com
%
% Original author:
% Xavier Danaux (xdanaux@gmail.com)
%
% License:
% CC BY-NC-SA 3.0 (http://creativecommons.org/licenses/by-nc-sa/3.0/)
%
% Important note:
% This template requires the moderncv.cls and .sty files to be in the same
% directory as this .tex file. These files provide the resume style and themes
% used for structuring the document.
%
%%%%%%%%%%%%%%%%%%%%%%%%%%%%%%%%%%%%%%%%%

%----------------------------------------------------------------------------------------
%	PACKAGES AND OTHER DOCUMENT CONFIGURATIONS
%----------------------------------------------------------------------------------------

\documentclass[10pt,a4paper,roman]{moderncv} % Font sizes: 10, 11, or 12; paper sizes: a4paper, letterpaper, a5paper, legalpaper, executivepaper or landscape; font families: sans or roman

\moderncvstyle{banking} % CV theme - options include: 'casual' (default), 'classic', 'oldstyle' and 'banking'
\moderncvcolor{black} % CV color - options include: 'blue' (default), 'orange', 'green', 'red', 'purple', 'grey' and 'black'
\usepackage[UTF8]{ctex}
%\usepackage{utf8x}{inputenc}
\usepackage{url}
\usepackage{lipsum} % Used for inserting dummy 'Lorem ipsum' text into the template
\usepackage[scale=0.8]{geometry} % Reduce document margins
%sugartom_scale=0.75->0.8
%add the following line by sugartom
\geometry{top=1.5cm,bottom=2.0cm,head=0.5cm}
%\setlength{\hintscolumnwidth}{3cm} % Uncomment to change the width of the dates column
%\setlength{\makecvtitlenamewidth}{10cm} % For the 'classic' style, uncomment to adjust the width of the space allocated to your name

%----------------------------------------------------------------------------------------
%	NAME AND CONTACT INFORMATION SECTION
%----------------------------------------------------------------------------------------

\firstname{Gaoxiong} % Your first name
\familyname{Zeng (曾高雄)} % Your last name

%----------------------------------------------------------------------------------------
\begin{document}

\makecvtitle % Print the CV title

%----------------------------------------------------------------------------------------
%	CONTACT SECTION
%----------------------------------------------------------------------------------------
\section{Contact Information}
\vspace{2mm}
\begin{tabular}{p{9cm} l}
Email: \href{mailto:gaoxiongzeng@qq.com}{gaoxiongzeng@qq.com} &  Office 3661 (Lift 31/32), Academic Building\\
Telephone: +852-68713173 & Hong Kong University of Science and Technology\\
Webpage: \textcolor{blue}{\url{http://gaoxiongzeng.github.io}} & Clear Water Bay, Hong Kong SAR, China\\
\end{tabular}

%----------------------------------------------------------------------------------------
%	EDUCATION SECTION
%----------------------------------------------------------------------------------------

\section{Education}
\vspace{2mm}
\cventry{2015 - Present}{Ph.D. in Computer Science and Engineering}{Hong Kong University of Science and Technology (HKUST)}{Hong Kong, China}{}{Advisor: Prof. Kai Chen}
\vspace{3mm}
\cventry{2011 - 2015}{B.E. in Electronic Engineering and Information Science}{University of Science and Technology of China (USTC)}{Hefei, China}{}{}


%----------------------------------------------------------------------------------------
%	EXPERIENCE SECTION
%----------------------------------------------------------------------------------------

\section{Experience}
%\cventry{Sept. 2019 - Present}{Research Intern. \textnormal{Worked with Prof. Kai Chen.}}
%{Peng Cheng Laboratory}{Shenzhen, China}{}{}

%\cventry{Aug. 2015 - Present}{Research and Teaching Assistant. \textnormal{Worked with Prof. Kai Chen, Dr. Cindy Li, etc.}}
%{Hong Kong University of Science and Technology}{Hong Kong, China}{}{}



\vspace{2mm}
\cvitemwithcomment{}{[Research Intern] {Peng Cheng Laboratory, Shenzhen, China}}{2019.09 - Present}
\vspace{1mm}
\cvitemwithcomment{}{[Research Intern] {Chinese Academy of Sciences, Beijing, China}}{2014.07 - 2014.08}

%----------------------------------------------------------------------------------------
%	INTERESTS SECTION
%----------------------------------------------------------------------------------------
\section{Research Interest}
\vspace{2mm}
Data center networking and machine learning systems (with particular interest in transport layer designs). 

%----------------------------------------------------------------------------------------
%	PUBLICATIONS SECTION
%----------------------------------------------------------------------------------------

%\section{Selected Research Projects}
%\cvitem{}{
%\begin{itemize}
%\item \textbf{XPath}, a simple, practical and readily deployable framework to implement explicit routing. XPath
%explicitly identifies an end-to-end path with a path ID and leverages a two-step compression algorithm to
%pre-install all the desired paths into IP TCAM tables of commodity switches.
%\end{itemize}
%}

\section{Publications}
\vspace{2mm}
\cvitem{Conference Publications}{
\vspace{1mm}
\begin{itemize}
\item \textbf{Congestion Control for Cross-Datacenter Networks} \\
\textbf{Gaoxiong Zeng}, Wei Bai, Ge Chen, Kai Chen, Dongsu Han, Yibo Zhu, Lei Cui \\
The 27th IEEE International Conference on Network Protocols (\textbf{ICNP}), 2019.
\end{itemize}
}

\vspace{1.5mm}

\cvitem{Journal Publications}{
\vspace{1mm}
\begin{itemize}
\item \textbf{Transport Protocols for Data Center Networks: A Survey / 数据中心网络传输协议综述} \\
\textbf{Gaoxiong Zeng}, Shuihai Hu, Junxue Zhang, Kai Chen \\
Journal of Computer Research and Development (\textbf{J-CRAD}), 2020.
\end{itemize}
}

\vspace{1.5mm}

\cvitem{Workshop Publications}{
\vspace{1mm}
\begin{itemize}
\item \textbf{Rethinking Transport Layer Design for Distributed Machine Learning} \\
Jiacheng Xia, \textbf{Gaoxiong Zeng}, Junxue Zhang, Weiyan Wang, Wei Bai, Junchen Jiang, Kai Chen \\
The 3rd ACM Asia-Pacific Workshop on Networking (\textbf{APNet}), 2019.
\item \textbf{Combining ECN and RTT for Datacenter Transport} \\
\textbf{Gaoxiong Zeng}, Wei Bai, Ge Chen, Kai Chen, Dongsu Han, Yibo Zhu \\
The 1st ACM Asia-Pacific Workshop on Networking (\textbf{APNet}), 2017.
\end{itemize}
}

\vspace{1.5mm}

\cvitem{Posters}{
\vspace{1mm}
\begin{itemize}
\item \textbf{Improved Path Compression for Explicit Path Control in Production Data Centers} \\
Shuihai Hu, Kai Chen, \textbf{Gaoxiong Zeng} \\
The 13th USENIX Symposium on Networked Systems Design and Implementation (\textbf{NSDI}), 2016.
\end{itemize}
}


%----------------------------------------------------------------------------------------
%	AWARDS SECTION
%----------------------------------------------------------------------------------------
\section{Honors \& Awards}
\vspace{2mm}
\cvitemwithcomment{}{[HKUST] {Research Postgraduate Scholarship}}{2015, 2016, 2017, 2018, 2019, 2020}
\vspace{1mm}
\cvitemwithcomment{}{[USTC] {Graduation Award of The Talent Program in Computer \& Information Sci. \& Tech.}}{2015}
\vspace{1mm}
\cvitemwithcomment{}{[USTC] {Outstanding Student Scholarship}}{2012, 2013, 2014}
\vspace{1mm}
\cvitemwithcomment{}{[Department of Education, Guangdong] {Provincial Soong Ching Ling Scholarship} }{2010}


\end{document}
